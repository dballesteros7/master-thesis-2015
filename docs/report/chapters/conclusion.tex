\chapter{Conclusion}
\label{sec:conclusion}

We have proposed two new models, FLDC and FFLDC, that extend the capabilities of a novel submodular probabilistic model, namely FLID. We have shown how the FLDC model can capture coherence in addition to diversity, and how FFLDC allows generalization through the use of features. Synthetic experiments show the performance of noise constrastive estimation for learning these models from data and the practical considerations that must be addressed such as how to avoid local maxima and the effect of various hyperparameters.

This type of models has been commonly used for document and image summarization, but we show how it can also be applied for recommending touristic locations using information extracted from an online image collection, namely Flickr. A complex probabilistic model such as FLDC captures higher order information resulting in a better approximation of the underlying data distribution.

Existing techniques for landmark identification allowed us to extract information and useful datasets from geo-tagged photos taken in the city of Zürich. Around 160,000 photos were collected representing perspectives of thousands of users, these were then translated into popular locations and user trips from which tourist behavior and routes were learned. For the large dataset, our best model was able to outperform the baselines that only model membership of elements in a set however the score was lower than for models that include information about the order of the elements, i.e. those that can model sequences.

Devising a model that naturally incorporates information about order of the elements is an interesting direction for future work. However, this could significantly increase the complexity of models which in its current form already make computations intractable due to the combinatorial nature of the set domains.

Spatial features of the photos were mainly used for the identification of landmarks and modeling the user's behavior, however more study is required to identify other features that can be used with the FFLDC model to generalize what can be learned from one city to another. Some experiments were performed in this direction but no conclusive results were obtained.

Despite the limitations, we believe that the proposed models can be applied to a wide range of problems because they can model complex distributions and identify characteristics about the data, such as item similarity.