\chapter{Related Work}
\label{sec:related_work}

Using online image collections to describe and present visual summarizes of places or cities has been recently a topic of interest. \citet{Kennedy2007, Kennedy2008} use photos from Flickr to produce visual summaries of locations and events, their analysis uses geographical labels, visual features and user-provided textual tags to identify representative photos of a location. Another approach was presented by \cite{Jaffe2006}, they apply clustering to the spatial features of photos and then evaluate these clusters based on number of photographers and textual tags to identify the most representative ones.

However, the scale of these works was limited to only single cities. \cite{Kleinberg2009} present a clustering approach to identifying landmarks and points of interest based on the number of geo-tagged photographs and photographers, this is done in scales that vary from a specific city to a whole country. Their work also show how visual features and textual tags can be used to infer the location of photo when geotags are not available. This is another research direction that addresses the problem that most of the photos uploaded to Flickr are not geotagged, for example \citet{Serdyukov2009} present a language-based model for predicting the most probable location in which a photo was taken based on the textual annotations provided by users. A different large scale approach to generating visual summaries of a place is presented by \citet{Li2009} where supervised learning, namely SVMs, is used instead of clustering, they also incorporate temporal information in addition to the visual and textual features.

Temporal information is an important feature when dealing with the problem of itineraries or touristic routes. \citet{Popescu2009} show how information such as travel and visit times between locations can be extracted from Flickr photos which then could be used to propose personalized trips with detailed schedules.

The methods presented by \cite{Kurashima2010} are closely related to our current approach, they propose the use of clustering techniques for landmark detection and a Markov chain for modeling photographers' behavior which can then be used to generate travel routes based on some previous history. Our contribution uses the same clustering techniques but introduces a novel model for the photographers' behavior.

This novel model is based on current research on submodular probabilistic models and summarization. Submodularity is desirable property of set functions and has been widely studied, see \cite{krause14submodular} for a survey. Much of the research is focused on inference \citep{djolonga14variational, djolonga15scalable} and sampling \citep{gotovos15sampling}. However, applications in document summarization have used submodularity to model aspects such as diversity and coverage. \citet{Shahaf2012, Shahaf2013} present a method for generating timelines representing structured summaries from large collections of documents, it shows how submodularity can be used to encourage diversity between documents. More closely related to the task of visual summarization, is the work presented by \citet{Tschiatschek2014}. They propose mixtures of submodular functions to generate summaries of large image collections and compare the performance of these automatically generated summaries to ones made by humans.
