\def\firstcircle{(0:-1.5cm) circle (2cm)}
\def\secondcircle{(90:2.5cm) circle (2cm)}
\def\thirdcircle{(0:1.5cm) circle (2cm)}

\pgfplotsset{colormap={blackwhite}{[2pt] % colormap steps: 2 pt
    % white: from 0000 to 1000
    rgb(0000pt)=(1.0,1.0,1.0);
    rgb(1000pt)=(1.0,1.0,1.0);
    % gray:  from 1000 to 2000
    rgb(1000pt)=(0.5,0.5,0.5);
    rgb(2000pt)=(0.5,0.5,0.5);
    % black: from 2000 to 3000
    rgb(2000pt)=(0.0,0.0,0.0);
    rgb(3000pt)=(0.0,0.0,0.0);
}}

% Now we can draw the sets:
\begin{tikzpicture}
  \definecolor{seriesA}{RGB}{222,235,247}
  \definecolor{seriesB}{RGB}{158,202,225}
  \definecolor{seriesC}{RGB}{49,130,189}
  \begin{scope}
      \begin{scope}[even odd rule]
          \clip \secondcircle (-3.5,-3.5) rectangle (3.5,3.5);
          \clip \thirdcircle (-3.5,-3.5) rectangle (3.5,3.5);
          \fill[seriesB] \firstcircle;
      \end{scope}
      \begin{scope}[even odd rule]
        \clip \firstcircle (-3.5,-3.5) rectangle (3.5,4.5);
        \clip \thirdcircle (-3.5,-3.5) rectangle (3.5,4.5);
        \fill[seriesB] \secondcircle;
      \end{scope}
      \begin{scope}[even odd rule]
        \clip \firstcircle (-3.5,-3.5) rectangle (3.5,3.5);
        \clip \secondcircle (-3.5,-3.5) rectangle (3.5,3.5);
        \fill[seriesB] \thirdcircle;
      \end{scope}
      \begin{scope}
        \clip \secondcircle;
        \fill[seriesC] \firstcircle;
      \end{scope}
      \begin{scope}
        \clip \thirdcircle;
        \fill[seriesC] \firstcircle;
      \end{scope}
      \begin{scope}
        \clip \thirdcircle;
        \fill[seriesA] \secondcircle;
      \end{scope}
      \begin{scope}
        \clip \firstcircle;
        \clip \thirdcircle;
        \fill[seriesA] \secondcircle;
      \end{scope}
      \draw \firstcircle node [below] {$A$};
      \draw \secondcircle node [above] {$B$};
      \draw \thirdcircle node [below] {$C$};
  \end{scope}

  \begin{axis}[
    hide axis,
    scale only axis,
    width=0pt,
    height=0pt
    colormap/blackwhite,
    colorbar,
    height=6cm,
    point meta min=0,
    point meta max=33,
    colorbar style={ytick={0,11.85, 32.2},at={(4.5cm,4.5cm)}}
    ]
    \addplot [draw=none] coordinates {(0,0)};
  \end{axis}
    % When using the above, you will notice that the border lines of the
    % original circles are erased by the intersection parts. To solve this
    % problem, either use a background layer (see the manual) or simply draw
    % the border lines after everything else has been drawn.
    
\end{tikzpicture}
